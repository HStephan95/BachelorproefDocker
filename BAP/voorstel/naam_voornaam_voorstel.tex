%==============================================================================
% Sjabloon onderzoeksvoorstel bachelorproef
%==============================================================================
% Gebaseerd op LaTeX-sjabloon ‘Stylish Article’ (zie voorstel.cls)
% Auteur: Jens Buysse, Bert Van Vreckem

\documentclass[fleqn,10pt]{voorstel}

%------------------------------------------------------------------------------
% Metadata over het voorstel
%------------------------------------------------------------------------------

\JournalInfo{HoGent Bedrijf en Organisatie}
\Archive{Bachelorproef 2017 - 2018} % Of: Onderzoekstechnieken

%---------- Titel & auteur ----------------------------------------------------

% TODO: geef werktitel van je eigen voorstel op
\PaperTitle{Docker for Windows Server}
\PaperType{Onderzoeksvoorstel Bachelorproef} % Type document

% TODO: vul je eigen naam in als auteur, geef ook je emailadres mee!
\Authors{Stephan Heirbaut\textsuperscript{1} Gert Schepens\textsuperscript{2}} % Authors
%\CoPromotor{Piet Pieters\textsuperscript{2}}
\affiliation{\textbf{Contact:}
  \textsuperscript{1} \href{mailto:stephan.heirbaut.w1409@student.hogent.be}{stephan.heirbaut.w1409@student.hogent.be};
  \textsuperscript{2} \href{mailto:schepens.gert@gmail.com}{schepens.gert@gmail.com};
}

%---------- Abstract ----------------------------------------------------------

\Abstract{ Docker for Windows Server 2016 is beschikbaar sinds september 2016, maar het heeft nog geen doorbraak gehad bij de DevOps. Dit ondanks het feit dat het een krachtige tool is voor hen, zeker als men ook een Microsoft Certified Partner wil zijn. Het voorstel van deze bachelorproef is om een onderzoek uit te voeren naar hoe krachtig Docker kan zijn op een Windows Server 2016, teneinde zo meer opties te hebben om deze technologie te deployen. Er zal dus een opstelling gebeuren van een Windows Server 2016 en een CentOS server met Docker, waarbij beiden de taak zullen krijgen om dezelfde applicaties te deployen. De verwachting is dat Docker het op beide platformen er even goed van af brengt, maar ook dat er zeker nog werk aan de winkel is op vlak van documentatie. DevOps is namelijk een groeiend principe, en terecht. Hoe meer opties zij dus hebben, hoe beter.
}

%---------- Onderzoeksdomein en sleutelwoorden --------------------------------
% TODO: Sleutelwoorden:
%
% Het eerste sleutelwoord beschrijft het onderzoeksdomein. Je kan kiezen uit
% deze lijst:
%
% - Mobiele applicatieontwikkeling
% - Webapplicatieontwikkeling
% - Applicatieontwikkeling (andere)
% - Systeembeheer
% - Netwerkbeheer
% - Mainframe
% - E-business
% - Databanken en big data
% - Machineleertechnieken en kunstmatige intelligentie
% - Andere (specifieer)
%
% De andere sleutelwoorden zijn vrij te kiezen

\Keywords{Systeem- en netwerkbeheer. Docker --- Windows Server 2016 --- Linux} % Keywords
\newcommand{\keywordname}{Sleutelwoorden} % Defines the keywords heading name

%---------- Titel, inhoud -----------------------------------------------------

\begin{document}

\flushbottom % Makes all text pages the same height
\maketitle % Print the title and abstract box
\tableofcontents % Print the contents section
\thispagestyle{empty} % Removes page numbering from the first page

%------------------------------------------------------------------------------
% Hoofdtekst
%------------------------------------------------------------------------------

% De hoofdtekst van het voorstel zit in een apart bestand, zodat het makkelijk
% kan opgenomen worden in de bijlagen van de bachelorproef zelf.
%---------- Inleiding ---------------------------------------------------------

\section{Introductie} % The \section*{} command stops section numbering
\label{sec:introductie}

Voorheen kon men Docker enkel installeren op Linux servers. Veel keuze had men dus niet wat betreft het OS waarop men het wou deployen. Maar recent werd Docker ook geïntroduceerd voor Windows Server 2016, waarin men handig gebruik maakt van de Hyper-V containers die er ingebouwd in zitten.
In de afgelopen 4 jaar waarin Docker op de markt is gekomen, heeft het veel gedaan voor DevOps-teams. Het is dus zeker de moeite waard om eens te kijken of de Hyper-V containers mooi integreren in Docker en hoe dat precies gebeurt.
Het doel en de onderzoeksvraag van deze bachelorproef is dus: Hoe vlot werkt Docker op Windows Server 2016, zeker voor het deployen van .Net-applicaties, ten opzichte van Docker voor Linux?

%---------- Stand van zaken ---------------------------------------------------

\section{State-of-the-art}
\label{sec:state-of-the-art}

Er werden reeds verschillende tests uitgevoerd met Docker, ~\autocite{Boettiger2014}, waarin het gebruikt werd om makkelijk reproduceerbaar onderzoek uit te voeren. In de conclusie worden verschillende best practices voorgesteld. Er zal in deze bachelorproef onderzocht worden in hoeverre het mogelijk is om deze na te bootsen op een Windows Server. Het uiteindelijke doel is om te proberen de volledige opstelling te repliceren.
Een andere studie, ~\autocite{Salman2016}, heeft ook al de voordelen besproken van het werken met containers op vlak van beveiliging, zoals bijvoorbeeld Sandboxing. Ook dit is een goede studie om te proberen reproduceren in de Windows omgeving, echter met een grotere uitdaging. In hun conclusie vermeldt men vooral cgroups en SELinux, functies die uniek voor Linux bestaan. Er zal dus onderzocht worden hoe deze nagebootst kunnen worden op een Windows Server.
Verder zal in deze bachelorproef de beschikbare literatuur ook grondig worden getest, aangezien Docker nog maar sinds september 2016 beschikbaar is voor Windows en dat deze misschien nog dient te worden bijgewerkt.
De belangrijkste officiële bronnen hiervoor zijn: ~\autocite{Docker2016}, ~\autocite{Friis2016} en ~\autocite{Container2016}. Deze zullen één voor één bekeken en getest worden op hun compleetheid.

%---------- Methodologie ------------------------------------------------------
\section{Methodologie}
\label{sec:methodologie}

Eerst en vooral zal er gestart worden met een literatuurstudie van de beschikbare informatie op de website en blogs van Docker. Daarbovenop zullen er ook verschillende onafhankelijke Ops-bronnen geraadpleegd worden, om na te gaan hoe zij hun Docker hebben opgesteld.
Vervolgens zullen er een Windows Server 2016 en een CentOS-server met Docker worden opgesteld en zal er vergeleken worden hoe goed beiden het doen in het opstellen van containers.
Als virtualisatie-platform zal VirtualBox gebruikt worden, waarbij de Windows Server opgesteld zal worden via Vagrant, Chocolatey en Powershell, en de CentOS met Vagrant en Ansible.

%---------- Verwachte resultaten ----------------------------------------------
\section{Verwachte resultaten}
\label{sec:verwachte_resultaten}

De verwachting is dat de Windows Server het op zijn minst even goed zal doen als Linux. Zeker op vlak van .Net-applicatie zou Windows met de Hyper-V containers geen problemen mogen hebben.
De benodigde resources voor het efficiënt draaien van de Windows Server zullen waarschijnlijk wel hoger liggen dan de Linux Server.

%---------- Verwachte conclusies ----------------------------------------------
\section{Verwachte conclusies}
\label{sec:verwachte_conclusies}

De documentatie voor de Windows Server zal wel moeilijker te verkrijgen zijn, aangezien de technologie hierop vrij nieuw is en de meeste mensen hiervoor nog steeds Linux gebruiken.
Echter de installatie zou vrij vlot moeten gaan via Powershell en Chocolatey. Beide zijn namelijk krachtige tools in hun eigen recht.



%------------------------------------------------------------------------------
% Referentielijst
%------------------------------------------------------------------------------
% TODO: de gerefereerde werken moeten in BibTeX-bestand ``voorstel.bib''
% voorkomen. Gebruik JabRef om je bibliografie bij te houden en vergeet niet
% om compatibiliteit met Biber/BibLaTeX aan te zetten (File > Switch to
% BibLaTeX mode)

\phantomsection
\printbibliography[heading=bibintoc]

\end{document}
