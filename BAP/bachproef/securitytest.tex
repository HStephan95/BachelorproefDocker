%%=============================================================================
%% Securitytest
%%=============================================================================

\chapter{Security test}
\label{ch:securitytest}

Dit hoofdstuk vergelijkt de verschillende manieren waarop beide besturingssystemen, Linux en Windows, in staat zijn om hun systemen te beschermen. Deze security review vergelijkt welke industriestandaarden beide systemen kunnen implementeren om hun systemen te harden tegen aanvallen.

Zoals eerder gezegd is een container automatisch veiliger dan een klassieke VM, omdat er een isolatielaag ligt tussen de container en het besturingssysteem waarop het draait. Echter, als men toch via de container toegang zou kunnen verkrijgen tot het host-systeem, wat dan? Elk systeem is namelijk even sterk als zijn zwakste schakel. Dit werd ook bevestigd voor Solomon Hykes, CTO van Docker op DockerCon 2017 in Austin, USA op vlak van Linux beveiliging. Hij zei: "We denken dat Docker niet de verantwoordelijkheid is om Linux subsystemen te beveiligen. Sterker nog, we denken niet dat deze verantwoordelijkheid bij één bedrijf in het bijzonder moet liggen. Linux is té groot en té belangrijk, dus daarom is het zó belangrijk dat veiligheid vanuit de community komt. En de community is goed bezig: iedereen werkt samen, in plaats van elk in zijn eigen hoekje".

Daarom zal deze security review vertrekken vanuit de populairste veiligheidsopties voor het harden van een CentOS machine, waarna deze vergeleken zullen worden met equivalente opties voor Windows.

\section{CentOS}
\subsection{Linux containers}
Als eerste werd er gekeken naar hoe Linux-containers tot staande gebracht worden en hoe sterk het isolatieniveau ervan is.

Zoals in de inleiding reeds vermeld hebben Linux-containers hun eigen netwerk resources, geheugen, geïsoleerde CPU en I/O block, maar maken ze wel gebruik van dezelfde kernel als het host besturingssysteem. Dit zorgt ervoor dat containers licht en gestroomlijnd kunnen zijn, maar dit kan zorgen voor zowel sterktes als zwaktes.

Om te beginnen is het veel moeilijker om diepe toegang te krijgen tot de hoogst mogelijke privileges bij een container. Dit is een direct gevolg van de isolatie tussen de container en zijn host. Echter, als dit wel lukt, is er een reëel probleem. De Linux Container deelt immers de kernel met zijn host-systeem, waardoor system calls een groot probleem vormen. Door een geminimaliseerde OS te gebruiken kan men dit probleem verminderen, maar niet verhelpen.

Daarnaast is er wel het voordeel dat Root-toegang tot de container niet automatisch betekent dat men Root-toegang heeft op het host-systeem, maar dan moet men wel oppassen voor privileged containers. Deze worden namelijk wel standaard gebonden aan het host uid 0, waardoor deze containers niet root-safe zijn en ook niet root-safe gemaakt kunnen worden. Daarom maakt men best gebruik van Mandatory Access Control, zoals SELinux of AppArmor, seccomp filters en het laten vallen van namespaces.

Vervolgens is er het probleem dat Cgroups standaard geen limiet heeft, waardoor men gemakkelijk een Denial of Service-attack kan uitvoeren op een CentOS-systeem. Dit risico kan verlaagd worden door memory, cpu en pods aan te passen in de configuratie van lxc.cgroup.

Ten slotte zijn aanvallers vaak teleurgesteld als ze in een container geraken, want best practice zegt "geen data in containers", waardoor de voornaamste aanvalreden niet veel zal opleveren. Kijk maar naar de ransomware attacks die steeds populairder worden. Deze hebben als doel om data 'gevangen' te nemen tot het losgeld ervoor betaald is. \autocite{Weissig2013 } \autocite{Reno2016} \autocite{Wang2017}

\subsection{SELinux}
SELinux of Security-Enhanced Linux gebruikt Linux Security Modules (LSM) om een Linux-systeem te harden tegen aanvallen. De verschillende modules kunnen worden aan- of uitgezet via een reeks van booleans. Daarnaast kan men ook handgemaakte waardes toevoegen aan deze lijst om ze te blokkeren of door te laten.

SELinux bestaat uit onderwerpen, zoals users of applicatieprocessen. Daarnaast bestaat het uit objecten, zoals bestanden en folders, en uit een set van regels. Deze regels bepalen wat een onderwerp mag doen met een object.

Voor Docker bestaat er ook een SEModule. Deze module moet eerst enabled worden via 'semodule -v -e docker'. Daarna dient Docker te worden herstart met de volgende waarde: --selinux-enabled. Deze waarde dient te worden geplaatst in docker.service configuratiebestand. Ten slotte dient men Docker opnieuw op te starten, waarna Docker altijd met SELinux enabled zal opstarten.

De invloed van deze module is veelvoudig. Het belangrijkste gevolg is dat men kan beperken tot welke folders Docker toegang heeft, want de 'privileged' Docker-processen krijgen niet dezelfde privileges als andere privileged processen. Daarnaast kan men via het 'docker run'-commando de kernel-mogelijkheden beperken door er '--cap-drop=' aan toe te voegen.

Van alle tools die beschikbaar zijn om een Linux-server te beveiligen, is SELinux veruit de krachtigste. Het grootste nadeel bij SELinux is echter dat de gebruiksvriendelijkheid niet evident is; het is een moeilijke tool. Ondanks dit nadeel loont het wel om vaardig te worden binnen deze module. \autocite{Henry-Stocker2012} \autocite{Juggery2017} \autocite{Walsh2013}

\subsection{Firewalld}
Daarnaast kan men op CentOS Firewalld configureren om bepaalde poorten te openen of te blokkeren. In essentie gebruikt Firewalld gewoon IpTables met een XML-overlay, wat het gebruik ervan veel vergemakkelijkt heeft. Hiermee kan men de verschillende zones en poorten instellen die men wil blokkeren of doorlaten, maar men kan ook gekende services toevoegen aan Firewalld, waarbij deze automatisch de standaardpoorten ervoor opent of sluit. Standaard is de zone 'public' en worden de meeste poorten geblokkeerd. \autocite{Brown2014} 

\subsection{Seccomp}
Ten slotte kan men ook Secure Computing gebruiken (Seccomp) gebruiken. Hiermee kan men de system calls beperken die een process kan maken. Hierdoor kan het systeem beschermd worden tegen hackers, wanneer ze system calls willen maken die niet eerder gedeclareerd zijn. Hoewel Seccomp in het verleden stroef was om te gebruiken, is het gebruik (Libseccomp) ervan sterk gegroeid, onder andere door de toevoeging van BPF (Berkeley Packet Filters). BPF werd in het verleden gebruikt om netwerk pakketten te filteren, maar door het potentieel ervan groeiden de toepassingen ervoor ook.

Seccomp's voordelen liggen vooral in het blokkeren van system calls vanuit de containers naar het host-systeem, als deze niet van toepassingen zijn. Daardoor kunnen aanvallen vanuit de container end-points tegengehouden worden. \autocite{Arora2012} \autocite{Spijkers2013}

\section{Windows}
\section{Hyper-V containers}
Hyper-V Containers, zoals die gebruikt worden door Docker for Windows, verschillen op één belangrijke manier van Linux Containers. Hyper-V containers gebruiken namelijk de base image om een VM aan te maken, door deze in de Hyper-V partition wrapper te steken, waarna er in deze VM een Windows-container wordt aangemaakt om de applicatie in te steken. Hierdoor is er een extra isolatielaag tussen de container en de kernel, namelijk de Hypervisor van de VM. Dit verhelpt veel van de problemen die de Linux Containers hebben op vlak van kwetsbaarheden.

Hier staat echter wel een kost tegenover. Dit zorgt er namelijk voor dat de Hyper-V containers in het algemeen groter zullen zijn dan Linux Containers, en dat ze meer tijd zullen nodig hebben om opgezet te worden. De last is gelukkig niet al te groot, aangezien Hyper-V een type 1 Hypervisor is en dus rechtstreeks op de hardware draait. Ten slotte bestaat er ook het gevaar dat de VM crasht wanneer men in de container zit, waardoor men niet meer uit de container boundaries zal geraken. \autocite{Savill2015}

\section{Integrity levels}
Windows Integrity Checks (WIC) is het Windows-equivalent van SELinux. Deze functie zat reeds ingebouwd in Windows Vista, waar het als doel had om objecten te beschermen, zoals bestanden, printers en pipes. WIC werkt door de betrouwbaarheid van alle objecten en de interactie hiermee een waarde te geven, zogezegd een level of trustworthiness. Als men een actie wil uitvoeren op dit object, zal men het niveau op zijn minst moeten evenaren. WIC heeft een grotere prioriteit dan normale permissies.

Er zijn in totaal zes niveaus die WIC kan geven aan objecten. Van laag naar hoog zijn dit:

\begin{itemize}[noitemsep]
	\item untrusted (anonieme processen)
	\item low (interactie met het internet)
	\item medium (meeste objecten, onder andere de shell)
	\item high (Administrators)
	\item system (system)
	\item installer (speciaal voor .EXE's)
\end{itemize}

Hoewel deze functie praktisch onopgemerkt is gebleven, heeft dit de veiligheid van Windows wel sterk verbeterd. Hiervoor had Windows namelijk geen enkele manier om aan Mandatory Access Control te doen. Helaas is er wel nog werk aan de winkel. Er is bijvoorbeeld een gebrek aan management of een configuratietool voor administrators. Daarnaast is het systeem ook niet feilloos, zoals er in Vyacheslav Rusakov wordt aangetoond. \autocite{Khanse2009} \autocite{Bradley2007} \autocite{Rusakov2016}

\section{Windows Defender}
Windows Defender is begonnen als een anti-spyware systeem, waarnaast men nog andere antivirus-functies moest installeren, zoals Microsoft Security Essentials, om een complete verdediging te hebben. Sinds Windows 8 heeft het echter ook definitief deze rol overgenomen.

Qua performantie krijgt Windows Defender gemengde reviews. Vooral de firewall van Windows Defender kan een last vormen op de CPU. Daardoor is de meest gezochte term voor Windows Defender hoe je het uitschakelt. Vaak is het beter om een extra third-party systeem te gebruiken voor de firewall, zoals een PFSense of een WatchGuard. \autocite{Hall2017} \autocite{Lefferts2017}

\subsection{ACL}
Het Windows-equivalent voor Seccomp is Windows Access Control Lists. De ACL bestaat uit een lijst van Access Control Entries (ACE's), waarbij elke ACE een trustee aanduidt en bepaalt welke Access Rights deze heeft. Helaas wordt het niet aangeraden om rechtstreeks met de inhoud van een ACL te werken. Er zijn functies en modules om deze te wijzigen, maar ook hier kan de administrator geen rechtstreekse invloed hebben op het systeem. \autocite{Rapid72017}

\section{Conclusie}
Linux heeft een meer hands-on aanpak om zijn systemen te beveiligen, waarbij de administrator van het Linux-systeem veel zelf kan en moet bepalen. Dit betekent wel dat de veiligheid van een Linux-opstelling even goed is als de kennis van zijn administrator, wat zowel positief als negatief kan zijn. Windows daarentegen neemt veel controle van de administrator weg, en doet het werk voor hem. Dit betekent dat er een basisniveau is qua veiligheid, maar dat dit niveau ver onder het niveau van een Linux-opstelling kan liggen, vooral als de administrator reeds veel kennis verworven heeft.