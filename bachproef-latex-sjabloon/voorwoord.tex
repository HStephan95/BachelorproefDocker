%%=============================================================================
%% Voorwoord
%%=============================================================================

\chapter*{Voorwoord}
\label{ch:voorwoord}

%% TODO:
%% Het voorwoord is het enige deel van de bachelorproef waar je vanuit je
%% eigen standpunt (``ik-vorm'') mag schrijven. Je kan hier bv. motiveren
%% waarom jij het onderwerp wil bespreken.
%% Vergeet ook niet te bedanken wie je geholpen/gesteund/... heeft

%De eerste zinnen van het voorwoord vertellen waar het boek, de scriptie of het verslag inhoudelijk over gaat.
%Daarna volgt de aanleiding tot het schrijven van het boek, de scriptie of het verslag.
%Vervolgens vertelt u uw ervaringen als schrijver van de geschreven tekst.
%U bedankt alle mensen die erbij betrokken waren en/of waar u mee heeft samengewerkt. Hierbij kunt u denken aan de vertaler, begeleider of proeflezer.
%Daarna bedankt u vrienden en familie voor de steun die zij u hebben gegeven tijdens het schrijven.
%Vervolgens ondertekent u het voorwoord met uw naam; het is tenslotte een persoonlijk stuk.
%U sluit af met de datum en de plaats.

