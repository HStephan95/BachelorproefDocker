%%=============================================================================
%% Voorwoord
%%=============================================================================

\chapter*{Voorwoord}
\label{ch:voorwoord}

%% TODO:
%% Het voorwoord is het enige deel van de bachelorproef waar je vanuit je
%% eigen standpunt (``ik-vorm'') mag schrijven. Je kan hier bv. motiveren
%% waarom jij het onderwerp wil bespreken.
%% Vergeet ook niet te bedanken wie je geholpen/gesteund/... heeft

%De eerste zinnen van het voorwoord vertellen waar het boek, de scriptie of het verslag inhoudelijk over gaat.
%Daarna volgt de aanleiding tot het schrijven van het boek, de scriptie of het verslag.
%Vervolgens vertelt u uw ervaringen als schrijver van de geschreven tekst.
%U bedankt alle mensen die erbij betrokken waren en/of waar u mee heeft samengewerkt. Hierbij kunt u denken aan de vertaler, begeleider of proeflezer.
%Daarna bedankt u vrienden en familie voor de steun die zij u hebben gegeven tijdens het schrijven.
%Vervolgens ondertekent u het voorwoord met uw naam; het is tenslotte een persoonlijk stuk.
%U sluit af met de datum en de plaats.

Deze bachelorproef voert een vergelijkende studie uit tussen Docker for Windows en Docker for Linux. Specifiek wordt er gebruik gemaakt van een Windows Server 2016 met GUI, waarop Docker for Windows wordt geïnstalleerd. Deze Linux omgeving bestaat uit een CentOS 7.4 waarop Docker for Linux wirdt geïnstalleerd. Op beide platformen worden daarna twee containers uitgerold: de eerste container is een Microsoft SQL server en de tweede container bevat een .NET webapplicatie.

Als eerste werd de documentatie vergeleken tussen beide omgevingen op vlak van volledigheid en gebruiksgemak.
Vervolgens werden beide systemen getest op de benodigde tijd voor het uitvoeren van een volledige installatie en de installatie van de containers. Voor de installatie van beide systemen wordt er gebruik gemaakt van Vagrant. Zodanig dat de installatie van het besturingssysteem, de benodigde applicaties en de containers zoveel mogelijk geautomatiseerd kon worden. Dit maakte het mogelijk om op een methodologische manier de performantie te testen van beide systemen.
Vervolgens werd er een vergelijkende studie uitgevoerd tussen de veiligheid van de CentOS-omgeving en de Windows-omgeving. Van Linux is geweten dat de veiligheid er goed is, als men alle functionaliteit ervan gebruikt. De vraag was dus: welke mogelijkheden heeft Windows om die te evenaren?
Ten slotte volgt er op het einde een conclusie welk systeem het best past binnen DevOps teams en hoe men het best met beide om gaat.

De reden waarom ik dit onderwerp koos zijn meervoudig: als eerste vind ik de Container-technologie uitermate fascinerend en ga ik hier zeker verder mee spelen na mijn bachelorproef. Daarnaast was ik aangenaam verrast dat Docker ging samenwerken met Microsoft om hun technologie beschikbaar te stellen voor het Windows OS. Dit past natuurlijk wel in het manifesto van Microsoft CEO Satya Nadella om het bedrijf een nieuwe richting te doen inslaan. Ten slotte leek me dit een uitermate geschikt onderwerp voor een bachelorproef. Er zijn al verschillende Blog posts en artikels verschenen hierover. Maar, de meeste raken alleen maar de oppervlakte van de zaak of erger: leveren geen concrete cijfers, maar alleen buikgevoel gedreven meningen.

De bachelorproef uitwerken ging vrij vlot. Buiten een paar problemen met de webapplicatie en het automatiseren van de Windows-omgeving zat dit wel snor. Het schrijven was een beetje moeilijker, aangezien er geen linguïst verloren is gegaan aan mij. Ondanks het moeilijke schrijven was het wel een aangename ervaring. Ik had nog niet eerder een werk van dit kaliber mogen opleveren en hoewel het zwaar werken was, ik ben wel tevreden van het werk dat ik geleverd heb.

Ten slotte zou ik nog graag enkele mensen bedanken:

Als eerste zou ik mijn promotor Steven Vermeulen willen bedanken. Hij heeft me vooral op literair vlak een stevige duw kunnen geven in de goede richting met zijn feedback.

Daarnaast zou ik ook graag mijn co-promotor willen bedanken: Gert Schepens. Zelfs tijdens drukke momenten heeft hij tijd kunnen vrijmaken voor mij en mijn bachelorproef.

Vervolgens wil ik ook even mijn vriendin in de bloemetjes zetten. Ze heeft mij meerdere malen geholpen bij het verbeteren van mijn teksten, wanneer ik wel wist wat ik wou schrijven. Maar, niet hoe.

Ten slotte wou ik ook Bert Van Vreckem bedanken voor zijn hulp bij dit onderwerp. Zonder zijn hulp zou de webapplicatie misschien nog steeds niet werken. Via hem ben ik ook op het idee gekomen om dit onderwerp te behandelen.

Stephan Heirbaut
25 mei 2018, Lokeren